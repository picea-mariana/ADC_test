\documentclass[]{article}
\usepackage{lmodern}
\usepackage{amssymb,amsmath}
\usepackage{ifxetex,ifluatex}
\usepackage{fixltx2e} % provides \textsubscript
\ifnum 0\ifxetex 1\fi\ifluatex 1\fi=0 % if pdftex
  \usepackage[T1]{fontenc}
  \usepackage[utf8]{inputenc}
\else % if luatex or xelatex
  \ifxetex
    \usepackage{mathspec}
  \else
    \usepackage{fontspec}
  \fi
  \defaultfontfeatures{Ligatures=TeX,Scale=MatchLowercase}
\fi
% use upquote if available, for straight quotes in verbatim environments
\IfFileExists{upquote.sty}{\usepackage{upquote}}{}
% use microtype if available
\IfFileExists{microtype.sty}{%
\usepackage{microtype}
\UseMicrotypeSet[protrusion]{basicmath} % disable protrusion for tt fonts
}{}
\usepackage[margin=1in]{geometry}
\usepackage{hyperref}
\hypersetup{unicode=true,
            pdftitle={Reproducible Papers with RMarkdown},
            pdfauthor={-Julia Stuart, jms2435@nau.edu -Bryce Mecum, mecum@nceas.ucsb.edu},
            pdfborder={0 0 0},
            breaklinks=true}
\urlstyle{same}  % don't use monospace font for urls
\usepackage{color}
\usepackage{fancyvrb}
\newcommand{\VerbBar}{|}
\newcommand{\VERB}{\Verb[commandchars=\\\{\}]}
\DefineVerbatimEnvironment{Highlighting}{Verbatim}{commandchars=\\\{\}}
% Add ',fontsize=\small' for more characters per line
\usepackage{framed}
\definecolor{shadecolor}{RGB}{248,248,248}
\newenvironment{Shaded}{\begin{snugshade}}{\end{snugshade}}
\newcommand{\KeywordTok}[1]{\textcolor[rgb]{0.13,0.29,0.53}{\textbf{#1}}}
\newcommand{\DataTypeTok}[1]{\textcolor[rgb]{0.13,0.29,0.53}{#1}}
\newcommand{\DecValTok}[1]{\textcolor[rgb]{0.00,0.00,0.81}{#1}}
\newcommand{\BaseNTok}[1]{\textcolor[rgb]{0.00,0.00,0.81}{#1}}
\newcommand{\FloatTok}[1]{\textcolor[rgb]{0.00,0.00,0.81}{#1}}
\newcommand{\ConstantTok}[1]{\textcolor[rgb]{0.00,0.00,0.00}{#1}}
\newcommand{\CharTok}[1]{\textcolor[rgb]{0.31,0.60,0.02}{#1}}
\newcommand{\SpecialCharTok}[1]{\textcolor[rgb]{0.00,0.00,0.00}{#1}}
\newcommand{\StringTok}[1]{\textcolor[rgb]{0.31,0.60,0.02}{#1}}
\newcommand{\VerbatimStringTok}[1]{\textcolor[rgb]{0.31,0.60,0.02}{#1}}
\newcommand{\SpecialStringTok}[1]{\textcolor[rgb]{0.31,0.60,0.02}{#1}}
\newcommand{\ImportTok}[1]{#1}
\newcommand{\CommentTok}[1]{\textcolor[rgb]{0.56,0.35,0.01}{\textit{#1}}}
\newcommand{\DocumentationTok}[1]{\textcolor[rgb]{0.56,0.35,0.01}{\textbf{\textit{#1}}}}
\newcommand{\AnnotationTok}[1]{\textcolor[rgb]{0.56,0.35,0.01}{\textbf{\textit{#1}}}}
\newcommand{\CommentVarTok}[1]{\textcolor[rgb]{0.56,0.35,0.01}{\textbf{\textit{#1}}}}
\newcommand{\OtherTok}[1]{\textcolor[rgb]{0.56,0.35,0.01}{#1}}
\newcommand{\FunctionTok}[1]{\textcolor[rgb]{0.00,0.00,0.00}{#1}}
\newcommand{\VariableTok}[1]{\textcolor[rgb]{0.00,0.00,0.00}{#1}}
\newcommand{\ControlFlowTok}[1]{\textcolor[rgb]{0.13,0.29,0.53}{\textbf{#1}}}
\newcommand{\OperatorTok}[1]{\textcolor[rgb]{0.81,0.36,0.00}{\textbf{#1}}}
\newcommand{\BuiltInTok}[1]{#1}
\newcommand{\ExtensionTok}[1]{#1}
\newcommand{\PreprocessorTok}[1]{\textcolor[rgb]{0.56,0.35,0.01}{\textit{#1}}}
\newcommand{\AttributeTok}[1]{\textcolor[rgb]{0.77,0.63,0.00}{#1}}
\newcommand{\RegionMarkerTok}[1]{#1}
\newcommand{\InformationTok}[1]{\textcolor[rgb]{0.56,0.35,0.01}{\textbf{\textit{#1}}}}
\newcommand{\WarningTok}[1]{\textcolor[rgb]{0.56,0.35,0.01}{\textbf{\textit{#1}}}}
\newcommand{\AlertTok}[1]{\textcolor[rgb]{0.94,0.16,0.16}{#1}}
\newcommand{\ErrorTok}[1]{\textcolor[rgb]{0.64,0.00,0.00}{\textbf{#1}}}
\newcommand{\NormalTok}[1]{#1}
\usepackage{longtable,booktabs}
\usepackage{graphicx,grffile}
\makeatletter
\def\maxwidth{\ifdim\Gin@nat@width>\linewidth\linewidth\else\Gin@nat@width\fi}
\def\maxheight{\ifdim\Gin@nat@height>\textheight\textheight\else\Gin@nat@height\fi}
\makeatother
% Scale images if necessary, so that they will not overflow the page
% margins by default, and it is still possible to overwrite the defaults
% using explicit options in \includegraphics[width, height, ...]{}
\setkeys{Gin}{width=\maxwidth,height=\maxheight,keepaspectratio}
\IfFileExists{parskip.sty}{%
\usepackage{parskip}
}{% else
\setlength{\parindent}{0pt}
\setlength{\parskip}{6pt plus 2pt minus 1pt}
}
\setlength{\emergencystretch}{3em}  % prevent overfull lines
\providecommand{\tightlist}{%
  \setlength{\itemsep}{0pt}\setlength{\parskip}{0pt}}
\setcounter{secnumdepth}{0}
% Redefines (sub)paragraphs to behave more like sections
\ifx\paragraph\undefined\else
\let\oldparagraph\paragraph
\renewcommand{\paragraph}[1]{\oldparagraph{#1}\mbox{}}
\fi
\ifx\subparagraph\undefined\else
\let\oldsubparagraph\subparagraph
\renewcommand{\subparagraph}[1]{\oldsubparagraph{#1}\mbox{}}
\fi

%%% Use protect on footnotes to avoid problems with footnotes in titles
\let\rmarkdownfootnote\footnote%
\def\footnote{\protect\rmarkdownfootnote}

%%% Change title format to be more compact
\usepackage{titling}

% Create subtitle command for use in maketitle
\newcommand{\subtitle}[1]{
  \posttitle{
    \begin{center}\large#1\end{center}
    }
}

\setlength{\droptitle}{-2em}

  \title{Reproducible Papers with RMarkdown}
    \pretitle{\vspace{\droptitle}\centering\huge}
  \posttitle{\par}
    \author{-``Julia Stuart,
\href{mailto:jms2435@nau.edu}{\nolinkurl{jms2435@nau.edu}}'' -``Bryce
Mecum,
\href{mailto:mecum@nceas.ucsb.edu}{\nolinkurl{mecum@nceas.ucsb.edu}}''}
    \preauthor{\centering\large\emph}
  \postauthor{\par}
      \predate{\centering\large\emph}
  \postdate{\par}
    \date{1/18/2019}


\begin{document}
\maketitle

Everything above the --- is in yaml (yet another markup language) You
could also use: author: {[}``Julia Stuart,
\href{mailto:jms2435@nau.edu}{\nolinkurl{jms2435@nau.edu}}'',``Bryce
Mecum,
\href{mailto:mecum@nceas.ucsb.edu\%22}{\nolinkurl{mecum@nceas.ucsb.edu"}}{]}

You can add a csl file to make it cite in a certain fashion i. e. Nature
csl: biomed\_central.csl (download a file and match it to the csl name
in the header)

\section{Abstract}\label{abstract}

There is a package to do something with citations: look at bib file
\texttt{citr} I really like using R (R Core Team 2015) for science
because of tools like RStudio (RStudio Team 2015) and RMarkdown
(RMarkdown Team 2015). This document is a quick demonstration of writing
an academic paper in RMarkdown. There's a lot of other resources
available on the web but hopefully you'll find this document useful as
an example.

\section{Introduction}\label{introduction}

Writing reports and academic papers is a ton of work but a large amount
of that work can be spent doing monotonous tasks such as:

\begin{itemize}
\tightlist
\item
  Updating figures and tables as we refine our analysis
\item
  Editing our analysis and, in turn, editing our paper's text
\item
  Managing bibliography sections and in-text citations/references
\end{itemize}

These monotonous tasks are also highly error-prone. With RMarkdown, we
can close the loop, so to speak, between our analysis and our manuscript
because the manuscript can become the analysis.

As an alternative to Microsoft Word, RMarkdown provides some advantages:

\begin{itemize}
\tightlist
\item
  Free to use
\item
  Uses text so we can:
\item
  Use version control for

  \begin{itemize}
  \tightlist
  \item
    Tracking changes
  \item
    Collaborating
  \end{itemize}
\item
  Edit it with our favorite and most powerful text editors
\item
  Use the command line to for automation
\end{itemize}

The rest of this document will show how we get some of the features we
need such as:

\begin{itemize}
\tightlist
\item
  Attractive typesetting for mathematics
\item
  Figures, tables, and captions
\item
  In-text citations
\item
  Bibliographies
\end{itemize}

\section{Methods}\label{methods}

Our analysis will be pretty simple. We'll use the \texttt{diamonds}
dataset from the \texttt{ggplot2} (Wickham 2009) package and run a
simple linear model. At the top of this document, we started with a code
chunk with \texttt{echo=FALSE} set as a chunk option so that we can load
the \texttt{ggplot2} package and \texttt{diamonds} dataset without
outputting anything to the screen.

For our analysis, we'll create a really great plot which really shows
the relationship between price and carat and shows how we include plots
in our document. Then we'll run a linear model of the form
\(y = mx + b\) on the relationship between price and carat and shows how
we include tables in our document. We can also put some more advanced
math in our paper and it will be beautifully typeset:

\[\sum_{i=1}^{N}{log(i) + \frac{\omega}{x}}\]

We can also use R itself to generate bibliographic entries for the
packages we use so we can give proper credit when we use other peoples'
packages in our analysis. Here we cite the \texttt{ggplot2} package:

\begin{Shaded}
\begin{Highlighting}[]
\OperatorTok{>}\StringTok{ }\KeywordTok{citation}\NormalTok{(}\StringTok{'ggplot2'}\NormalTok{)}

\NormalTok{To cite ggplot2 }\ControlFlowTok{in}\NormalTok{ publications, please use}\OperatorTok{:}

\StringTok{  }\NormalTok{H. Wickham. ggplot2}\OperatorTok{:}\StringTok{ }\NormalTok{Elegant Graphics }\ControlFlowTok{for}\NormalTok{ Data Analysis. Springer}\OperatorTok{-}\NormalTok{Verlag New York, }\DecValTok{2009}\NormalTok{.}

\NormalTok{A BibTeX entry }\ControlFlowTok{for}\NormalTok{ LaTeX users is}

  \OperatorTok{@}\NormalTok{Book\{,}
\NormalTok{    author =}\StringTok{ }\NormalTok{\{Hadley Wickham\},}
\NormalTok{    title =}\StringTok{ }\NormalTok{\{ggplot2}\OperatorTok{:}\StringTok{ }\NormalTok{Elegant Graphics }\ControlFlowTok{for}\NormalTok{ Data Analysis\},}
\NormalTok{    publisher =}\StringTok{ }\NormalTok{\{Springer}\OperatorTok{-}\NormalTok{Verlag New York\},}
\NormalTok{    year =}\StringTok{ }\NormalTok{\{}\DecValTok{2009}\NormalTok{\},}
\NormalTok{    isbn =}\StringTok{ }\NormalTok{\{}\DecValTok{978}\OperatorTok{-}\DecValTok{0}\OperatorTok{-}\DecValTok{387}\OperatorTok{-}\DecValTok{98140}\OperatorTok{-}\DecValTok{6}\NormalTok{\},}
\NormalTok{    url =}\StringTok{ }\NormalTok{\{http}\OperatorTok{:}\ErrorTok{//}\NormalTok{ggplot2.org\},}
\NormalTok{  \}}
\end{Highlighting}
\end{Shaded}

And then we just place that in our \texttt{.bibtex} file.

\section{Results}\label{results}

\begin{figure}

{\centering \includegraphics{paper-example_files/figure-latex/pricevscarat-1} 

}

\caption{Figure  1: The relationship between price and carat for the diamonds dataset.}\label{fig:pricevscarat}
\end{figure}

But the model was even better:

\begin{longtable}[]{@{}lrrrr@{}}
\caption{Table 1: This is a broomed linear model summary
table.}\tabularnewline
\toprule
term & estimate & std.error & statistic & p.value\tabularnewline
\midrule
\endfirsthead
\toprule
term & estimate & std.error & statistic & p.value\tabularnewline
\midrule
\endhead
(Intercept) & -2256.36 & 13.06 & -172.83 & 0\tabularnewline
carat & 7756.43 & 14.07 & 551.41 & 0\tabularnewline
\bottomrule
\end{longtable}

We were delighted to find that the slope parameter was 7756.43.

\section{Discussion}\label{discussion}

This was just a quick demonstration of a reproducible paper that
combined text, analysis, figures, tables, and citations into multiple
output formats (HTML, PDF). Hopefully you found it useful.

A lot of people are using RMarkdown these days so there are tons of
resources online but here are a few choice ones specifically about
making papers:

\begin{itemize}
\tightlist
\item
  \url{http://rmarkdown.rstudio.com/authoring_bibliographies_and_citations.html}
\item
  \url{http://svmiller.com/blog/2016/02/svm-r-markdown-manuscript/}
\item
  \url{http://www.petrkeil.com/?p=2401}
\end{itemize}

\section{References}\label{references}

No matter which citation manager you use you can get it to export into
.bib Which is text, which is best

\hypertarget{refs}{}
\hypertarget{ref-RCoreTeam}{}
R Core Team. 2015. ``R: A Language and Environment for Statistical
Computing.'' \url{http://www.r-project.org}.

\hypertarget{ref-RMarkdown}{}
RMarkdown Team. 2015. \emph{Rmarkdown: R Markdown Document Conversion, R
Package}. Boston, MA: RStudio, Inc. \url{http://rmarkdown.rstudio.com/}.

\hypertarget{ref-RStudio}{}
RStudio Team. 2015. \emph{RStudio: Integrated Development Environment
for R}. Boston, MA: RStudio, Inc. \url{http://www.rstudio.com/}.

\hypertarget{ref-ggplot}{}
Wickham, Hadley. 2009. \emph{Ggplot2: Elegant Graphics for Data
Analysis}. Springer-Verlag New York. \url{http://ggplot2.org}.


\end{document}
